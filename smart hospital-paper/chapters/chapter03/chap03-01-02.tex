%
% File: chap03-01-02.tex
% Author: 
% Description: Front-end Tools
%

\paragraph{Figma}\mbox{}\\

In our project, our goal is to design a website for medical use. As a team, we all agree that a high-quality website should include several essential factors. Among them, user interface (UI) and user experience (UX) are the most crucial. A well-designed website should maintain a consistent and visually appealing style across all pages, while also being intuitive to use and free from confusing workflows. Another key consideration is the design process itself. Since we are working as team, it is important to choose a tool and workflow that support efficient collaboration. First, team members should be allowed to work independently or collaboratively while still ensuring consistency across design. Second, we need a real-time collaborative environment to prevent any individual’s progress from being delayed by others. Therefore, Figma is our primary choice for designing the prototype of our website.

Figma is cloud-based design and prototyping tool that enables real-time collaboration among team members. It allows us to efficiently create UI, UX, wireframes, mockups and interactive prototype. These features allow our team to collaborate through the cloud. Meanwhile, we can work independently on individual tasks without interfering with one another’s progress. From a broader perspective, Figma enable us to review each other’s work instantly, ensuring our overall workflow and visual consistency remain coherent across all pages. One of the most practical benefits is that Figma can automatically generate CSS code for layout, fonts, and other design properties. This greatly enhances our efficiency when translating designs into front-end code. For those reasons, we choose Figma as the primary tool for prototyping our website.

\paragraph{HTML}\mbox{}\\

We aim to provide a clean, accessible and responsive interface for medical users. To achieve this goal, we requires tools that help us build structured, semantic layouts while maintaining visual consistency across devices and screen sizes. When evaluating which front-end tools were most suitable for our design, we considered several combinations, including HTML with Bootstrap, HTML with TailwindCSS, and SCSS-based modular design systems. According to the specific characteristics of our project and our situation. We all agree that using HyperText Markup Language(HTML) and Cascading Style Sheets(CSS) was the most appropriate choice for our needs after discussion. In the following section, we provide a brief introduction to these two technologies and explain how they contributed to the development of our prototype.

HTML is a necessary and standard core language that can be interpreted by any modern web browsers. In web development, HTML is used to arrange the content of the website and provide basic structural syntax for page design. As a fundamental tool, HTML has a relatively simple and easy-to-learn syntax, allowing developers to create websites with clear semantic structures that enhances both accessibility and user-friendliness. HTML wild ranges of labels, such as <head>, <p>, and <div>, which are used to organise the layout of a page. These elements support essential functions like hyperlinks, image embedding, list creation and tables formatting. With just basic syntax, developers can easily achieve a certain level of responsive functionality. Although various tools exists to assist in web development—such as page builders or frameworks—all of them ultimately compile down to HTML. As a result, HTML is irreplaceable in the front-end development process. Due to the situation our team has encountered, we all agree that the cost of learning alternative tools would beyond their benefits. Therefore, we decide to directly use plain HTML as our primary front-end tool. However, as mentioned above, HTML only provides basic structure and limiting styling capabilities. Managing layout and appearance directly within HTML would lead to inefficiencies and inconsistency. To address this issue, we employed CSS to handle visual design and layout, which will be discussed in next section. 

\paragraph{CSS}\mbox{}\\

One crucial issue in web development is how to manage code efficiently and make it maintainable. A widely adopted and effective solution is to separate the content, functionality and the style into distinct layers. While tools such as SCSS, Tailwind CSS and Bootstrap provide advanced styling features, we ultimately chose to use standard CSS for this project. This decision was based on the simplicity and clarity that CSS provides--especially valuable for a small team working on a focused prototype. CSS allows us to directly translate our Figma designs into code without introducing additional syntax or dependencies. This made it easier for all team members to understand and contribute to the styling process. Besides, one of the key advantages of CSS is its clear separation of content and presentation, which improves maintainability and enables us to update visual styles without modifying the fundamental structure. CSS also plays a crucial role in transforming our HTML structure into a clean, professional, and user-friendly interface. Furthermore, CSS enhances our collaboration by allowing us to define a shared layout and style system. Each members can easily apply consistent styles across pages, significantly reducing time spent on detail adjustments, and improving the efficiency of our workflow.
