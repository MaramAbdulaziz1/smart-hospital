%
% File: chap03.tex
% Author:
% Description: Design and Implementations
%
\chapter{Design and Implementations}
\label{chap:D&I}

Begins a chapter. Example: When the beloved cellist (Christopher Walken - outstanding) of a world-renowned string quartet receives a life-changing diagnosis, the group's future suddenly hangs in the balance: suppressed emotions, competing egos and uncontrollable passions threaten to derail years of friendship and collaboration. Featuring a brilliant ensemble cast (including Philip Seymour Hoffman, Catherine Keener and Mark Ivanir as the three other quartet members), it is a fascinating look into the world of working musicians, and an elegant homage to chamber music and the cultural world of New York. The music, of course, is ravishing (the score is the work of regular David Lynch collaborator Angelo Badalamenti): A Late Quartet hits all the right notes.

\section{Methodology}
\label{sec:sec01}

\subsection{Work Flow}
\label{subsec:subsec01}

\subsection{Process}
\label{subsec:subsec02}

\subsection{Front-end Tools}
\label{subsec:subsec03}

\subsection{Back-end Tools}
\label{subsec:subsec04}

\subsection{Testing Tools}
\label{subsec:subsec05}

\section{Front-end Design and Implementations}
\label{sec:sec02}

\subsection{Early Stage-Virtual Hospital Africa(VHA)}
\label{subsec:subsec01}

During the initial discussions with the client, it was identified that the \textbf{Patient Intake page}
contained too many information fields. This led to poor user experience, with many testers
abandoning the form midway (Figure~\ref{fig:intake1}).
To address this, the client designed a second version of a multi-step form interface (Figure~\ref{fig:intake2}).
Finally, after our team's feedback, a third version was created (Figure~\ref{fig:intake3}).

\subsubsection{Key Improvements in Patient Intake Page}
\begin{enumerate}
    \item \textbf{Left-side navigation bar:} Enabled non-linear navigation, allowing users to jump to
    any section freely, improving flexibility and control.
    \item \textbf{Optimized field grouping:} Consolidated fields such as name, gender, title, language,
    and ID number into a ``General'' group. Added profile photo upload with instant preview.
    \item \textbf{Unified navigation buttons:} Replaced inconsistent buttons with clear
    \emph{Back/Next}, improving predictability and reducing disorientation.
\end{enumerate}

\subsubsection{Patient Profile Page Improvements}
\begin{enumerate}
    \item \textbf{Enhanced patient information card:} Added profile picture, gender/age tags, patient status,
    quick contact and download buttons, and last edit time.
    \item \textbf{Expanded information modules:} Unified the top tab bar and added a secondary menu
    (General, Primary Care, Contacts, Biometrics, Insurance).
    \item \textbf{Editable forms:} Personal data presented in structured fields supporting direct editing.
\end{enumerate}

\subsubsection{Vitals Page Redesign}
\begin{enumerate}
    \item \textbf{Grouped layout:} Divided vital signs into ``Required'' and ``Optional'' sections,
    aligning with clinical practices.
    \item \textbf{Optimized workflow:} Replaced single ``Continue'' button with \emph{Back/Next},
    allowing users better control.
    \item \textbf{Integration with patient panel:} Displayed patient summary and treatment timeline
    alongside data entry.
\end{enumerate}

These improvements adhered to usability principles such as
\emph{Match between system and the real world},
\emph{Visibility of system status}, and
\emph{Recognition rather than recall},
reducing cognitive load, improving efficiency, and minimizing errors.



\subsection{Smart Hospital}
\label{subsec:subsec02}

\section{Back-end Design and Implementations}
\label{sec:sec03}
