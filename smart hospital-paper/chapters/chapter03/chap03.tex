%
% File: chap03.tex
% Author:
% Description: Design and Implementations
%
\chapter{Design and Implementations}
\label{chap:D&I}

Begins a chapter. Example: When the beloved cellist (Christopher Walken - outstanding) of a world-renowned string quartet receives a life-changing diagnosis, the group's future suddenly hangs in the balance: suppressed emotions, competing egos and uncontrollable passions threaten to derail years of friendship and collaboration. Featuring a brilliant ensemble cast (including Philip Seymour Hoffman, Catherine Keener and Mark Ivanir as the three other quartet members), it is a fascinating look into the world of working musicians, and an elegant homage to chamber music and the cultural world of New York. The music, of course, is ravishing (the score is the work of regular David Lynch collaborator Angelo Badalamenti): A Late Quartet hits all the right notes.

\section{Methodology}
\label{sec:sec01}

\subsection{Work Flow}
\label{subsec:subsec01}

\subsection{Process}
\label{subsec:subsec02}

\subsection{Front-end Tools}
\label{subsec:FETools}
%
% File: chap03-01-02.tex
% Author: 
% Description: 
% 3.1 Methodology
%  3.1.2Front-end Tools
%

\paragraph{Figma}\mbox{}\\

In our project, our goal is to design a website for medical use. As a team, we all agree that a high-quality website should include several essential factors. Among them, user interface (UI) and user experience (UX) are the most crucial. A well-designed website should maintain a consistent and visually appealing style across all pages, while also being intuitive to use and free from confusing workflows. Another key consideration is the design process itself. Since we are working as team, it is important to choose a tool and workflow that support efficient collaboration. First, team members should be allowed to work independently or collaboratively while still ensuring consistency across design. Second, we need a real-time collaborative environment to prevent any individual’s progress from being delayed by others. Therefore, Figma is our primary choice for designing the prototype of our website.

Figma is cloud-based design and prototyping tool that enables real-time collaboration among team members. It allows us to efficiently create UI, UX, wireframes, mockups and interactive prototype. These features allow our team to collaborate through the cloud. Meanwhile, we can work independently on individual tasks without interfering with one another’s progress. From a broader perspective, Figma enable us to review each other’s work instantly, ensuring our overall workflow and visual consistency remain coherent across all pages. One of the most practical benefits is that Figma can automatically generate CSS code for layout, fonts, and other design properties. This greatly enhances our efficiency when translating designs into front-end code. For those reasons, we choose Figma as the primary tool for prototyping our website.

\paragraph{HTML}\mbox{}\\

We aim to provide a clean, accessible and responsive interface for medical users. To achieve this goal, we requires tools that help us build structured, semantic layouts while maintaining visual consistency across devices and screen sizes. When evaluating which front-end tools were most suitable for our design, we considered several combinations, including HTML with Bootstrap, HTML with TailwindCSS, and SCSS-based modular design systems. According to the specific characteristics of our project and our situation. We all agree that using HyperText Markup Language(HTML) and Cascading Style Sheets(CSS) was the most appropriate choice for our needs after discussion. In the following section, we provide a brief introduction to these two technologies and explain how they contributed to the development of our prototype.

HTML is a necessary and standard core language that can be interpreted by any modern web browsers. In web development, HTML is used to arrange the content of the website and provide basic structural syntax for page design. As a fundamental tool, HTML has a relatively simple and easy-to-learn syntax, allowing developers to create websites with clear semantic structures that enhances both accessibility and user-friendliness. HTML wild ranges of labels, such as \verb|<head>|, \verb|<p>|, and \verb|<div>|, which are used to organise the layout of a page. These elements support essential functions like hyperlinks, image embedding, list creation and tables formatting. With just basic syntax, developers can easily achieve a certain level of responsive functionality. Although various tools exists to assist in web development—such as page builders or frameworks—all of them ultimately compile down to HTML. As a result, HTML is irreplaceable in the front-end development process. Due to the situation our team has encountered, we all agree that the cost of learning alternative tools would beyond their benefits. Therefore, we decide to directly use plain HTML as our primary front-end tool. However, as mentioned above, HTML only provides basic structure and limiting styling capabilities. Managing layout and appearance directly within HTML would lead to inefficiencies and inconsistency. To address this issue, we employed CSS to handle visual design and layout, which will be discussed in next section. 

\paragraph{CSS}\mbox{}\\

One crucial issue in web development is how to manage code efficiently and make it maintainable. A widely adopted and effective solution is to separate the content, functionality and the style into distinct layers. While tools such as SCSS, Tailwind CSS and Bootstrap provide advanced styling features, we ultimately chose to use standard CSS for this project. This decision was based on the simplicity and clarity that CSS provides--especially valuable for a small team working on a focused prototype. CSS allows us to directly translate our Figma designs into code without introducing additional syntax or dependencies. This made it easier for all team members to understand and contribute to the styling process. Besides, one of the key advantages of CSS is its clear separation of content and presentation, which improves maintainability and enables us to update visual styles without modifying the fundamental structure. CSS also plays a crucial role in transforming our HTML structure into a clean, professional, and user-friendly interface. Furthermore, CSS enhances our collaboration by allowing us to define a shared layout and style system. Each members can easily apply consistent styles across pages, significantly reducing time spent on detail adjustments, and improving the efficiency of our workflow.

\paragraph{Spring Boot and Thymeleaf}\mbox{}\\

Before selecting our technology stack for the Smart Hospital web application, we weighed a number of backend and frontend framework possibilities with consideration of equating development speed, maintainability, and integration efficiency. The top contenders included Spring Boot with Thymeleaf, Node.js with Express and React, and Django with HTML templates. Table 1\textbf{(@TODO:haven't applied yet)} summarizes the strengths and weaknesses of each stack based on thoroughly documented pros and cons according to credible sources.

We weighed these options and opted to go with Spring Boot in the backend and Thymeleaf in the frontend. This Java-focused stack provided us with a consistent and streamlined development environment, which was particularly crucial for our small team of individuals working on tight deadlines. Because both frameworks are based on the Java ecosystem, we were able to build full-stack functionality in one integrated project. This reduced the complexity that would have otherwise been caused by having two separate frontend and backend codebases and improved coordination among team members [3].\textbf{(@TODO:haven't applied yet)}

Spring Boot is a well-used enterprise-level framework that streamlines Java web application development. It provides an embedded Tomcat server, auto-configuration, and dependency management, allowing us to concentrate on core functionality instead of boilerplate setup [1]. Its innate RESTful API support enabled us to plan a scalable, modular backend architecture, with Controller classes as the basis for API endpoints, a core feature for connecting the client interface to backend services.

At the frontend, Thymeleaf was selected as a server-side templating engine that has native support for Spring Boot's MVC framework. Its dynamic binding of backend data in HTML templates ensured tight coupling of presentation logic with backend processes. This eliminated the need for extra JavaScript frameworks like React or Angular, reducing the learning curve and allowing the team to progress faster. Thymeleaf's human-readable syntax also made collaboration easier and the codebase more maintainable [2]\textbf{(@TODO:haven't applied yet)}.

Alternatives such as Node.js with Express and React were also considered, which would have offered highly interactive UIs and would have leveraged JavaScript's extensive ecosystem, but would also have meant two codebases to manage and deployment complexity [3]. Django with HTML templates, knowingly able to achieve rapid development and secure by design [1]\textbf{(@TODO:haven't applied yet)}, was also a consideration but would have required the team to learn Python, an undesirable learning curve considering the time constraints we had.

Finally, our academic supervisor recommended picking up Spring Boot with Thymeleaf for its combination of performance, integration ease, and suitability for a Java-experienced team. This also simplified deployment, as both backend and frontend could be deployed together in a single application. Moving further to the backend implementation phase, Spring Boot Controller classes will accept incoming HTTP requests and will form the basis of our API layer, which, later on, will be tested by tools discussed in the Backend Tools section.


\subsection{Back-end Tools}
\label{subsec:subsec04}

\subsection{Testing Tools}
\label{subsec:subsec05}

\section{Front-end Design and Implementations}
\label{sec:sec02}

\subsection{Early Stage-Virtual Hospital Africa(VHA)}
\label{subsec:subsec01}

During the initial discussions with the client, it was identified that the \textbf{Patient Intake page}
contained too many information fields. This led to poor user experience, with many testers
abandoning the form midway (Figure~\ref{fig:intake1}).
To address this, the client designed a second version of a multi-step form interface (Figure~\ref{fig:intake2}).
Finally, after our team's feedback, a third version was created (Figure~\ref{fig:intake3}).

\subsubsection{Key Improvements in Patient Intake Page}
\begin{enumerate}
    \item \textbf{Left-side navigation bar:} Enabled non-linear navigation, allowing users to jump to
    any section freely, improving flexibility and control.
    \item \textbf{Optimized field grouping:} Consolidated fields such as name, gender, title, language,
    and ID number into a ``General'' group. Added profile photo upload with instant preview.
    \item \textbf{Unified navigation buttons:} Replaced inconsistent buttons with clear
    \emph{Back/Next}, improving predictability and reducing disorientation.
\end{enumerate}

\subsubsection{Patient Profile Page Improvements}
\begin{enumerate}
    \item \textbf{Enhanced patient information card:} Added profile picture, gender/age tags, patient status,
    quick contact and download buttons, and last edit time.
    \item \textbf{Expanded information modules:} Unified the top tab bar and added a secondary menu
    (General, Primary Care, Contacts, Biometrics, Insurance).
    \item \textbf{Editable forms:} Personal data presented in structured fields supporting direct editing.
\end{enumerate}

\subsubsection{Vitals Page Redesign}
\begin{enumerate}
    \item \textbf{Grouped layout:} Divided vital signs into ``Required'' and ``Optional'' sections,
    aligning with clinical practices.
    \item \textbf{Optimized workflow:} Replaced single ``Continue'' button with \emph{Back/Next},
    allowing users better control.
    \item \textbf{Integration with patient panel:} Displayed patient summary and treatment timeline
    alongside data entry.
\end{enumerate}

These improvements adhered to usability principles such as
\emph{Match between system and the real world},
\emph{Visibility of system status}, and
\emph{Recognition rather than recall},
reducing cognitive load, improving efficiency, and minimizing errors.



\subsection{Smart Hospital}
\label{subsec:subsec02}
%
% File: chap03-02-01.tex
% Author: 
% Description: 
% 3.2 Front-end Design & Implementation
%  3.2.1 Design Process
%  front-end Implementation
% TODO:something wrong in chapter title
%

\paragraph{Design Process}\mbox{}
\paragraph{From Concept to User Needs: Ideation and Story Mappingigma}\mbox{}\\

\paragraph{Bridging Stories and Systems: Workflow Design}\mbox{}\\

\paragraph{Visualizing Functionality: Prototyping with Figma}\mbox{}\\
\textbf{(@TODO:need oganised)}

After we completed the workflow diagram, we used it as the foundation to divide tasks among team members. Each member is responsible for designing specific pages for different parts of the website, and we continually review each other’s work using the comment feature to ensure real-time feedback and update.\textbf{(TODO:need picture)} To evaluate whether the designs meet our requirements, we test various user scenarios using Figma’s prototyping feature to link different steps and assess the overall flow. For example, we add a “view” button on the patient profile page to allow doctors to access records of past visits, which helps them make more informed diagnoses.\textbf{(TODO:need picture)}

Color selection plays a vital role in website design. In our project, we choose a range of blue tones as the primary color scheme, as blue often represents inclusiveness and trust. To make the overall atmosphere feel more approachable and friendly, we incorporate some earthy tones into the blue. Meanwhile, white and gray is used for elements like buttons and text areas to create a clean and orderly appearance. Through these color combinations, we aim to convey a sense of stability and warmth to our users. The overall visual design reflects the spirit of our platform–to serve as a strong and trustworthy support for users of all ages, genders, physical conditions and cultural background.\textbf{(TODO:need picture to present the style)}

Next, we will introduce our layout design. The primary goal of our layout is to ensure intuitive user interaction. For example when doctors are diagnosing a patient, they typically start by reviewing the patient’s profile to gain a comprehensive understanding of the patient’s condition. This information is displayed on the “Patient Profile” page.\textbf{(TODO:need picture)} At the same time, doctors may need access to the patient’s visits history, so we design a clear and concise list that displays the date of each visit, a brief summary of the diagnosis, and the attending physician.\textbf{(TODO:need picture)} If more detailed information is needed, doctors can click the “view” button to open a pop-up window showing the full report.\textbf{(TODO:need picture)} Another example is during the diagnosis process–doctors may want to take temporary notes before writing the final prescription or summary. To support this, we included a text area on the “Clinical Notes” page where doctors can write their notes down.\textbf{(TODO:need picture)} These features are all designed to enhance the experience for medical professionals interacting with our platform.



\paragraph{Front-end Implementation}\mbox{}\\
\textbf{(@TODO:need oganised)}

In the front-end implementation section, we illustrate the process of how we created, organised, and tested our front-end code. We divide this process into 5 stages. First, we describe how we structured and developed the codebase. Then, we explain how we integrate shared layouts using Thymeleaf to improve consistency and maintainability across pages. Next, we explain how different pages are connected to form a smooth user flow. We also discuss the steps we took to test each function to ensure the website behaves as intended. Finally, we reflect on the main challenges we encountered during the front-end development and how we addressed them. These topics are covered in following subsections: Code Structure and Component Organisation, Integrating Shared Layouts with Thymeleaf, Page Navigation and User Flow, User Flow and Functional Testing, and Challenges Encountered During Frontend Development, respectively.
\paragraph{Code Structure and Component Organisation}\mbox{}\\

In this section, we describe how we constructed the front-end codebase. As mentioned earlier, each member was responsible for designing different parts of the website in Figma. We continued the pattern during implementation–each member developed their assigned parts using HTML and CSS, focusing on accurately reproducing the layout and visual style from our Figma design.

At this stage, our goal is to replicate the layout and visual style of our Figma design. Most interactive elements, such as buttons, pop-up windows, and form submissions, have not yet been functionalized. We have also not yet integrated a shared layout system using Thymeleaf. These shared elements are currently duplicated across individual pages,\textbf{(TODO:need picture)} and we plan to organize them in the next development stage to improve maintainability and consistency.

\paragraph{Integrating Shared Layouts with Thymeleaf}\mbox{}\\

After developing all the pages, we noticed that many structural elements–such as headers, sidebars, and forms–were repeated across different pages. To reduce redundancy and improve maintainability, we started integrating a shared layout system using Thymeleaf.

We created a base template that includes the common layout structure and used Thymeleaf’s th:replace and th:fragment features to inject shared components into each page.\textbf{(TODO:need picture)} This allows individual pages to focus only on their unique content while inheriting the same visual framework.
\paragraph{Page Navigation and User Flow}\mbox{}\\

In this stage, we began to functionalize interactive elements such as buttons, hyperlinks to ensure the website provides a smooth and intuitive experience. The overall user flow had already been defined in our early-stage Figma design, a clear user flow that reflects typical interactions between patients and doctors in our early-stage Figma design, reflecting typical interactions between patients and doctors. Each page was connected according to the logical steps users would take when navigating through the platform. For example, clicking the “Register” button on the homepage directs users to the registration form, and buttons on the “Dashboard” page are linked to corresponding pages. These workflows were clearly visualized in our Figma prototype.\textbf{(TODO:need picture)} The main goal of this stage is to ensure that all the interactive elements function as the origin design. In the next stage, we plan to test the whole website work appropriately and collect the disadvantages that can be improved.
\paragraph{User Flow and Functional Testing}\mbox{}\\

\paragraph{Challenges Encountered During Frontend Development}\mbox{}\\

One of the biggest challenges encountered while performing frontend development was how to break down the Figma-based design system into Thymeleaf templates. Even though the Figma designs created a solid visual reference for layout, typography, and spacing between elements, the server-side rendering approach of Thymeleaf required the interface to be split into smaller, more manageable parts. It introduced additional difficulty in the process of converting pixel-perfect designs into coherent HTML templates with alignment, responsiveness, and style consistency preserved. Visual parity usually needed to be delivered by iterative adjustments of the HTML structure as well as accompanying CSS.

An analogous problem involved the handling of shared styles and bits within the multi-page Thymeleaf application. Shared interface pieces, the sidebar navigation, header, and footer, were included as reusable fragments to foster consistency and reduce redundancy. With this came, however, the risk of side effects: a modification of a shared fragment would likely disrupt the styling or layout of multiple pages. To tackle this, the development team implemented a formalized update process, such as fragment-specific testing and peer review prior to merge of changes. This procedure ensured global component updates maintained visual correctness without causing regressions in other parts of the system.


\section{Back-end Design and Implementations}
\label{sec:sec03}
