\paragraph{Programming Language}\mbox{}\\
There are three main programming languages we considered for the backend: Java, Python, and C. Most of us had just finished the \textit{Object-Oriented Programming with Java} module~\cite{uob-oop-java-2024}, so Java is familiar and quick to pick up for us. C is a high-performance language, and we had also learned it in our first-semester \textit{Programming in C} course~\cite{uob-prog-in-c-2024}, but manual memory management and the lack of higher-level web frameworks would slow us down and increase maintenance risk. The last language considered was Python. It is good for fast prototyping, but our experience was uneven—some team members had never even used it before. Under time pressure, it would be quite challenging for all of us to learn and effectively use it.

We therefore chose Java with Spring Boot. Java's static typing and compile-time checks help us model clinical data more safely. Spring provides built-in support for REST endpoints and data access, and Maven keeps builds simple. This stack let us deliver features steadily within the course schedule.
