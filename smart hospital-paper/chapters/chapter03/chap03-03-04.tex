\paragraph{Implementation}\mbox{}\\
To implement the database, we converted the conceptual design into a set of SQL queries, available at \url{https://github.com/MaramAbdulaziz1/smart-hospital/tree/main/conf/database.sql}. The entire relational structure is defined in \texttt{database.sql}, ensuring relationships are enforced through primary keys, foreign keys, and appropriate uniqueness constraints. All tables use the InnoDB engine to guarantee transactional consistency and referential integrity. Data types were selected to balance correctness and performance: \texttt{char(36)} for UUIDs, \texttt{datetime} for timestamps, and \texttt{tinyint} for enumerated values (e.g., roles, departments, gender). Default values were set for audit fields such as \texttt{create\_time} and \texttt{update\_time}, and unique constraints prevent duplication in critical identifiers (e.g., emails, employee IDs, patient codes).

\mbox{}\\
Table creation followed the logical dependency graph. The \texttt{User} table was created first as the base referenced by most others. In parallel, \texttt{Doctor}, \texttt{Nurse}, \texttt{Patient}, and \texttt{HealthQA} were created, each referencing \texttt{User} via \texttt{user\_id}. Next, \texttt{Appointment} was defined using \texttt{patient\_id} and \texttt{provider\_id} (a user linked to a clinician). Downstream entities—\texttt{NurseVital}, \texttt{PatientIntake}, and \texttt{Prescription}—reference \texttt{Appointment}, and \texttt{Medication} links to \texttt{Prescription}. This modular decomposition supports scalability and maintainability by keeping domain logic clearly separated across entities. For testing, instead of seeding static SQL fixtures, we inserted mock data through the frontend forms. During development resets, tables were manually cleared and refreshed using DBeaver, our preferred MySQL GUI.
