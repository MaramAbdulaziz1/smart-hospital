\subsection{System Overview}
\label{subsec:reservation-overview}

According to section 3.4.1, a medical platform can choose traditional and dynamic interactive systems. Traditional form-based technology usually leads to lower user participation and a relatively low completion rate of reservations. Therefore, we chose to implement a system based on RESTful APIs and an instant, dynamic, and interactive updates system at the end.

The whole process of implementation includes three main phases: First, integrate spring-boot-starter-web and spring-boot-starter-security dependencies; Second, build RESTful API endpoints; Lastly, before operating the interface, add AJAX functionality on the frontend to support dynamic interactions with Thymeleaf templates.

\subsection{Appointment Lifecycle and Status Management}
\label{subsec:appointment-lifecycle}

The appointment system implements the whole life cycle, which includes four different statuses. The system supports two types of reservations: doctor clinic and nurse vitals check, and it focuses on different designs' appointment rules and authorisation.

Appointment status cycle:
\begin{itemize}
    \item Upcoming (Initial Status): Patient creates an appointment $\rightarrow$ next step: cancelled
    \item Complete (First Status): Appointment has been completed
    \item Cancelled (Second Status): Appointment could be cancelled by the patients
    \item Expired (Final Status): Appointment has expired due to a past date
\end{itemize}

Status transition rules:
\begin{itemize}
    \item Upcoming status can be transformed to cancelled
    \item Cancelled is the final status, can't be transformed again
    \item Doctors can view their appointments on the dashboard
\end{itemize}

\subsection{Booking Process and Conflict Prevention}
\label{subsec:booking-process}

To ensure appointment accessibility, this system designs a complete conflict prevention mechanism. Before reservation confirmation, the system will implement multiple authorisation checks:

Appointment process:
\begin{itemize}
    \item Patient authorisation check: use @AuthenticationPrincipal to authorise the patient role and login status.
    \item Database constraint validation: prevent double booking through unique key constraints.
    \item uk\_patient\_time: prevents the same patient from booking multiple appointments at the same time
    \item uk\_provider\_time: prevents the same provider from having multiple appointments at the same time
\end{itemize}

The code below showcases the conflict observation appointment flows:

\begin{lstlisting}[language=Java, caption=Appointment Booking Implementation]
@PostMapping("/book")
@PreAuthorize("hasRole('PATIENT')")
public BaseResponse<String> bookAppointment(
    @AuthenticationPrincipal UserDetails userDetails,
    @RequestBody AppointmentBook appointmentBook) {
    return appointmentService.book(userDetails.getUsername(), appointmentBook);
}
\end{lstlisting}

Mapping MyBatis SQL insert is below:

\begin{lstlisting}[language=SQL, caption=Appointment Insert Query]
<insert id="insertAppointment">
    INSERT INTO Appointment
    (appointment\_id, patient\_id, provider\_id, `date`, appoint\_time, status)
    VALUES
    (uuid(), \#{patientId}, \#{providerId}, \#{date}, \#{appointTime}, \#{status})
</insert>
\end{lstlisting}

\subsection{Cancellation Function}
\label{subsec:cancellation-function}

The cancellation system allows a patient to cancel an appointment after authorisation, and then the system updates the appointment status to cancelled.

Cancellation flow:
\begin{itemize}
    \item Authorisation check: check that the user who has the right to cancel.
    \item Status check: check that the appointment is in a cancellable state.
    \item Status update: update appointment status to cancelled.
\end{itemize}

Database operations through MyBatis implementation, and use the optimised SQL queries and UUID primary key to support the system scalability:

\begin{lstlisting}[language=SQL, caption=Appointment Status Update Query]
<update id="updateStatusByPatientId">
    UPDATE Appointment
    SET status = \#{status}
    WHERE
        appointment\_id = \#{appointmentId}
        AND patient\_id = \#{patientId}
        AND status = 0
</update>
\end{lstlisting}

This appointment system ensures a reliable appointment process through database restrictions and provides a seamless user experience while maintaining data integrity and system scalability.

