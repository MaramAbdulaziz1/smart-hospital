\subsection{System Overview}
\label{subsec:reservation-overview}

According to section 3.4.1, a medical platform can choose traditional and dynamic interactive systems. Traditional form-based technology usually leads to lower user participation and a relatively low completion rate of reservations. Therefore, we chose to implement a system based on RESTful APIs and an instant, dynamic, and interactive updates system at the end.

The whole process of implementation includes three main phases: First, integrate spring-boot-starter-web and spring-boot-starter-security dependencies; Second, build RESTful API endpoints; Lastly, before operating the interface, add AJAX functionality on the frontend to support dynamic interactions with Thymeleaf templates.

\subsection{Technical Implementation}
\label{subsec:reservation-technical}

Spring-boot-starter-web offers Spring MVC and embedded web server functionality~\cite{spring-mvc}, which must be added as a dependency in the project, in order to construct the Web ability of the Spring Boot application. Spring-boot-starter-security then provides authentication and check function~\cite{spring-security}. The whole reservation system operates with four core components: patient authentication, reservation API, run-time active calendar, and complete reserve life-cycle management. Based on safety considerations, Spring Security requires operators to implement the right management that is correctly based on role-based control and session management, not just based on basic authorisation. The following code demonstrates the implementation of role-based access control:

\begin{lstlisting}[language=Java, caption=Role-based Access Control Implementation]
@PreAuthorize("hasRole('PATIENT')")
public BaseResponse<List<AppointmentRecord>> getPatientUpcoming(
    @AuthenticationPrincipal UserDetails userDetails) {
    return appointmentService.getPatientUpcoming(userDetails.getUsername());
}
\end{lstlisting}

\subsection{Database Operations}
\label{subsec:reservation-database}

This reservation system includes multiple critical functions, including two appointment types (doctor consultations and vital signs checkups), dynamic department and doctor choice, real-time form validation, and complete observation history check. The system, through MyBatis, operates databases and uses optimised SQL queries based on UUID-based primary keys to support scalability~\cite{mybatis}. The database operations are implemented using MyBatis, as shown below:

\begin{lstlisting}[language=SQL, caption=Appointment Insert Query]
<insert id="insertAppointment">
    INSERT INTO Appointment
    (appointment\_id, patient\_id, provider\_id, `date`, appoint\_time, status)
    VALUES
    (uuid(), \#{patientId}, \#{providerId}, \#{date}, \#{appointTime}, \#{status})
</insert>
\end{lstlisting}

The complete function was implemented in a GitHub repository, with comprehensive appointment booking, management, and calendar features.

