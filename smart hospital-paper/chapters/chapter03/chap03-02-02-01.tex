%
% File: chap03-02-01.tex
% Author: 
% Description: 
% 3.2 Front-end Design & Implementation
%  3.2.1 Design Process
%  front-end Implementation
% TODO:something wrong in chapter title
%

\paragraph{Design Process}\mbox{}
\paragraph{From Concept to User Needs: Ideation and Story Mappingigma}\mbox{}\\

\paragraph{Bridging Stories and Systems: Workflow Design}\mbox{}\\

\paragraph{Visualizing Functionality: Prototyping with Figma}\mbox{}\\
\textbf{(@TODO:need oganised)}

After we completed the workflow diagram, we used it as the foundation to divide tasks among team members. Each member is responsible for designing specific pages for different parts of the website, and we continually review each other’s work using the comment feature to ensure real-time feedback and update.\textbf{(TODO:need picture)} To evaluate whether the designs meet our requirements, we test various user scenarios using Figma’s prototyping feature to link different steps and assess the overall flow. For example, we add a “view” button on the patient profile page to allow doctors to access records of past visits, which helps them make more informed diagnoses.\textbf{(TODO:need picture)}

Color selection plays a vital role in website design. In our project, we choose a range of blue tones as the primary color scheme, as blue often represents inclusiveness and trust. To make the overall atmosphere feel more approachable and friendly, we incorporate some earthy tones into the blue. Meanwhile, white and gray is used for elements like buttons and text areas to create a clean and orderly appearance. Through these color combinations, we aim to convey a sense of stability and warmth to our users. The overall visual design reflects the spirit of our platform–to serve as a strong and trustworthy support for users of all ages, genders, physical conditions and cultural background.\textbf{(TODO:need picture to present the style)}

Next, we will introduce our layout design. The primary goal of our layout is to ensure intuitive user interaction. For example when doctors are diagnosing a patient, they typically start by reviewing the patient’s profile to gain a comprehensive understanding of the patient’s condition. This information is displayed on the “Patient Profile” page.\textbf{(TODO:need picture)} At the same time, doctors may need access to the patient’s visits history, so we design a clear and concise list that displays the date of each visit, a brief summary of the diagnosis, and the attending physician.\textbf{(TODO:need picture)} If more detailed information is needed, doctors can click the “view” button to open a pop-up window showing the full report.\textbf{(TODO:need picture)} Another example is during the diagnosis process–doctors may want to take temporary notes before writing the final prescription or summary. To support this, we included a text area on the “Clinical Notes” page where doctors can write their notes down.\textbf{(TODO:need picture)} These features are all designed to enhance the experience for medical professionals interacting with our platform.



\paragraph{Front-end Implementation}\mbox{}\\
\textbf{(@TODO:need oganised)}

In the front-end implementation section, we illustrate the process of how we created, organised, and tested our front-end code. We divide this process into 5 stages. First, we describe how we structured and developed the codebase. Then, we explain how we integrate shared layouts using Thymeleaf to improve consistency and maintainability across pages. Next, we explain how different pages are connected to form a smooth user flow. We also discuss the steps we took to test each function to ensure the website behaves as intended. Finally, we reflect on the main challenges we encountered during the front-end development and how we addressed them. These topics are covered in following subsections: Code Structure and Component Organisation, Integrating Shared Layouts with Thymeleaf, Page Navigation and User Flow, User Flow and Functional Testing, and Challenges Encountered During Frontend Development, respectively.
\paragraph{Code Structure and Component Organisation}\mbox{}\\

In this section, we describe how we constructed the front-end codebase. As mentioned earlier, each member was responsible for designing different parts of the website in Figma. We continued the pattern during implementation–each member developed their assigned parts using HTML and CSS, focusing on accurately reproducing the layout and visual style from our Figma design.

At this stage, our goal is to replicate the layout and visual style of our Figma design. Most interactive elements, such as buttons, pop-up windows, and form submissions, have not yet been functionalized. We have also not yet integrated a shared layout system using Thymeleaf. These shared elements are currently duplicated across individual pages,\textbf{(TODO:need picture)} and we plan to organize them in the next development stage to improve maintainability and consistency.

\paragraph{Integrating Shared Layouts with Thymeleaf}\mbox{}\\

After developing all the pages, we noticed that many structural elements–such as headers, sidebars, and forms–were repeated across different pages. To reduce redundancy and improve maintainability, we started integrating a shared layout system using Thymeleaf.

We created a base template that includes the common layout structure and used Thymeleaf’s th:replace and th:fragment features to inject shared components into each page.\textbf{(TODO:need picture)} This allows individual pages to focus only on their unique content while inheriting the same visual framework.
\paragraph{Page Navigation and User Flow}\mbox{}\\

In this stage, we began to functionalize interactive elements such as buttons, hyperlinks to ensure the website provides a smooth and intuitive experience. The overall user flow had already been defined in our early-stage Figma design, a clear user flow that reflects typical interactions between patients and doctors in our early-stage Figma design, reflecting typical interactions between patients and doctors. Each page was connected according to the logical steps users would take when navigating through the platform. For example, clicking the “Register” button on the homepage directs users to the registration form, and buttons on the “Dashboard” page are linked to corresponding pages. These workflows were clearly visualized in our Figma prototype.\textbf{(TODO:need picture)} The main goal of this stage is to ensure that all the interactive elements function as the origin design. In the next stage, we plan to test the whole website work appropriately and collect the disadvantages that can be improved.
\paragraph{User Flow and Functional Testing}\mbox{}\\

\paragraph{Challenges Encountered During Frontend Development}\mbox{}\\

One of the biggest challenges encountered while performing frontend development was how to break down the Figma-based design system into Thymeleaf templates. Even though the Figma designs created a solid visual reference for layout, typography, and spacing between elements, the server-side rendering approach of Thymeleaf required the interface to be split into smaller, more manageable parts. It introduced additional difficulty in the process of converting pixel-perfect designs into coherent HTML templates with alignment, responsiveness, and style consistency preserved. Visual parity usually needed to be delivered by iterative adjustments of the HTML structure as well as accompanying CSS.

An analogous problem involved the handling of shared styles and bits within the multi-page Thymeleaf application. Shared interface pieces, the sidebar navigation, header, and footer, were included as reusable fragments to foster consistency and reduce redundancy. With this came, however, the risk of side effects: a modification of a shared fragment would likely disrupt the styling or layout of multiple pages. To tackle this, the development team implemented a formalized update process, such as fragment-specific testing and peer review prior to merge of changes. This procedure ensured global component updates maintained visual correctness without causing regressions in other parts of the system.
