%
% File: chap01.tex
% Author: Victor F. Brena-Medina
% Description: Introduction chapter where the biology goes.
%
\chapter{Introduction}
\label{chap:intro}

%=========================================================

\section{Motivation}
\label{sec:sec01}

The program will build a virtual hospital website called Smart Hospital, which aims to provide better access to medical care for residents in remote or medically underserved areas in Africa. Due to geographical limitations, transportation difficulties, and insufficient medical facilities, many residents often cannot receive good medical care. Through this virtual hospital website, residents in these areas will have the opportunity to consult with specialists remotely and receive more medical assistance and health support.

\noindent {This project was initially part of a collaboration with the Africa Virtual Hospital initiative, which is dedicated to addressing healthcare disparities in remote and underserved communities across Africa. The initiative aims to connect these underserved regions with doctors and experts from around the world through its virtual hospital platform. However, despite the termination of the collaboration due to certain factors, our team chose to independently continue this project. In the context of rising global healthcare pressures, including an aging population, increasing rates of chronic diseases, and uneven distribution of medical resources, we recognize that establishing a digital healthcare website to provide telemedicine remains highly meaningful, particularly for regions with insufficient medical resources or limited access to healthcare.}

\section{Client Brief}
\label{sec:sec02}

This project was initially undertaken in collaboration with Africa Virtual Hospital. As the client, Africa Virtual Hospital proposed the development of a virtual healthcare software platform comprising the following five main components: Doctor Application, WhatsApp Chatbots for Patients and Service Providers, Caregiver Application, Nurse Application, and Electronic Health Record. Within this framework, our team's primary focus was on the doctor application, which was expected to run on multiple devices, including smartphones, tablets, and laptops.

\noindent {The Doctor Application component was expected to include the following elements:}

\begin {itemize}
    \item Registration
    \item SSO/Welcome
    \item Home page (Appointments and Reviews)
    \item Calendar (Availability)
    \item Notifications
    \item Patient Profile
    \item Connect (Video and Voice to Patient WhatsApp)
    \item Treatment Plan and Orders
    \item Patient Information Feed
    \item My Supporting Staff
    \item My Profile
\end {itemize}

\noindent {During the collaboration period, we developed the various webpages and features of the doctor application based on the design and technical resources provided by the client. However, with the termination of the collaboration, our team also lost the original technical support. And then our new client is our University, but given the constraints on development time and resources, our team adjusted the scope based on existing resources and the original proposal, re-planned the website architecture and front-end design, and prioritized retaining core functional modules such as the doctor-patient interaction interface and basic medical information viewing, to ensure the project's feasibility under current conditions.}

\section{Aims and Objectives}
\label{sec:sec03}

\noindent {The aim of this project is to design and develop an easy-to-use, cross-device virtual hospital platform to promote telemedicine and doctor-patient communication, thereby improving medical accessibility in remote areas.}

\noindent {Based on the aim, we have established the following specific implementation items:}

\noindent {Develop an intuitive user interface and create a web-based system that supports physician operations. Implement core system functions, including appointment scheduling, patient record access, and basic remote consultation workflows. Establish an integrated patient data interface to help physicians understand the patient's overall condition, thereby improving diagnostic efficiency and decision-making quality.}

\section{Challenges}
\label{sec:sec04}

This project faced three main challenges: unexpected termination of cooperation, limited development time, and difficulties in obtaining back-end data resources.

\noindent {First, the project began in June and was expected to end in early September. Under the original plan, we were to collaborate with Africa Virtual Hospital to complete the development of the Doctor Application. Although most team members lacked website development experience, the development work proceeded smoothly under the guidance of Africa Virtual Hospital in the early stages.}

\noindent {However, the partnership was terminated prematurely in mid-July, resulting in the team losing access to technical support and shared data, and significantly compressing the remaining development timeline. With only about one and a half months left for development, the team had to independently complete tasks such as system design and construction, leading to an extremely tight schedule.}

\noindent {Additionally, the team had originally planned to utilize existing data from Africa Virtual Hospital, such as patient basic information and medical records. However, due to the termination of the collaboration, our backend development team faced difficulties in accessing patient data and medical records, which are highly sensitive, thereby imposing some limitations on website functionality verification and data simulation.}

\noindent {Despite these challenges, our team adjusted the project scope, reallocated resources, and prioritized the implementation of core functions to ensure the platform's feasibility.}

%=========================================================